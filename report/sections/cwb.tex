\newpage{}\section{CWB}

\subsection{Utilizzo del WorkBench}

CWB (\emph{Concurrency WorkBench}) è uno strumento di analisi statica creato
per scopi accademici al fine di verificare diversi tipi di equivalenze e
proprietà.

Questo strumento è stato pensato principalmente per sistemi le cui
implementazioni sono definite con il CCS e forniscono la possibilità di
verificare delle proprietà espresse per mezzo della HML (\emph{Hennessy-Milner
Logic}).

Grazie alla successione di comandi presente in sezione \ref{sec:cwb-definition}
il workbench viene popolato con i sistemi visti in sezione
\ref{sec:ccs-model}.

Viene definita una specifica \textbf{Spec} tale che riesca a distinguere il
corretto alternarsi di messaggi con numero di sequenza 0 e 1:

$ > Spec = = ok_0.ok_1.Spec $

Vengono aggiunti anche due nuovi agenti e sistemi, distinguibili dal prefisso
\textbf{Loud}. In tali sistemi:

\begin{itemize}
  \item ogni volta che viene inviato un messaggio con numero di sequenza X
    da un \textbf{Sender}, viene effettuata una comunicazione sul canale
    \textbf{$send_X$};
  \item ogni volta che viene ricevuto un messaggio con numero di sequenza X
    da un \textbf{Receiver} o da un \textbf{BadReceiver}, viene effettuata una
    comunicazione sul canale \textbf{$receive_X$}.
\end{itemize}

\subsection{Verifica di proprietà}

A questo punto può essere utile vedere se il nostro sistema soddisfa delle
proprietà.

Per popolare il workbench e testare le proprietà desiderabili per questo,
vengono usati comandi estratti dal listato presente in sezione
\ref{sec:cwb-commands}.

\subsubsection{Presenza di deadlock}

Può essere utile sapere se i sistemi (sia quello affidabile che quello lossy)
possono andare in deadlock in qualche esecuzione. Dopo l'esecuzione del comando
\texttt{deadlocks(}$<Sys>$\texttt{)}, il workbench ci informa che non sono
presenti deadlock.

In questo caso il risultato rispecchia le attese, poichè:

\begin{enumerate}
  \item già in \textbf{pseuCo} si poteva notare che uno dei due sistemi non
    soffriva di deadlock;
  \item il protocollo in sè prevede che i messaggi vengano trasmessi
    continuamente, senza considerare situazioni in cui uno dei due nodi
    della rete si spegne o interruzioni di qualsiasi natura.
\end{enumerate}

\subsubsection{Bisimilarità forte}

Subito dopo essersi accertati dell'assenza di deadlock, è utile conoscere se i
sistemi individuati hanno un comportamento equivalente a quello della
specifica \textbf{Spec} ($Sys \sim Spec$).

Come è facile aspettarsi, nessuno dei due sistemi è fortemente bisimile alla
specifica. Tuttavia questo non stupisce, poichè nei due ABP vengono eseguite
delle azioni interne che la specifica non è capace di
eseguire\footnote{infatti con il comando
\texttt{dfstrong(}$<Sys>$\texttt{,Spec)} il workbench informa che il primo
``sa compiere'' un'azione interna \texttt{tau}}.

Nemmeno i due sistemi sono fortemente bisimili tra loro: il sistema lossy è
distinguibile a causa di una sequenza di azioni interne che prevede il rifiuto
del messaggio inviato. A causa di questo, non è vero che dopo cinque azioni
interne è sempre possibile vedere la comunicazione \texttt{$ok_0$}.

\subsubsection{Bisimilarità debole}

I due sistemi, però, ci si aspetta che siano debolmente bisimili alla
specifica poichè come visto prima l'unica differenza tra questi è la presenza
di azioni interne che permettono il corretto funzionamento del protocollo.

Dopo l'esecuzione dei comandi per tale verifica, il workbench conferma, come
aspettato, che i due sistemi e la specifica \textbf{Spec} precedentemente
definita sono debolmente bisimili.

\subsubsection{Proprietà HML}

Vengono verificate, sempre con l'ausilio dello stesso tool, un insieme di
proprietà per ogni sistema:

\paragraph{ReliableABP} \mbox{}

Per il sistema senza perdita di messaggi vengono provate le seguenti proprietà:

\begin{itemize}
  \item $Can(tau)$: vero -- il sistema può eseguire un'azione iniziale
    (interna)
  \item $Even(<ok_0>T)$: vero -- il sistema esegue sempre $ok_0$ in
    un'esecuzione completa
  \item $Even(<ok_1>T)$: vero -- il sistema esegue sempre $ok_1$ in
    un'esecuzione completa
  \item $Even([[ok_0]]<ok_1>T) \&{} Even([[ok_1]]<ok_0>T)$: vero -- in
    un'esecuzione completa, il sistema:
    \begin{itemize}
      \item per ogni modo in cui può eseguire $ok_0$, poi può eseguire $ok_1$;
      \item per ogni modo in cui può eseguire $ok_1$, poi può eseguire $ok_0$.
    \end{itemize}
  \item $Fair$: vero -- è invariante che in qualsiasi modo venga eseguito $ok$
    per un certo messaggio di sequenza, poi verrà eseguito quello per il
    messaggio con il numero di sequenza successivo;
  \item $Pos(<ok_0><ok_0>T | <ok_1><ok_1>T)$ falso -- viene testato che il
    \textbf{Receiver} non possa ricevere due messaggi consecutivi con lo stesso
    numero di sequenza;
  \item $NoConsecutiveSameNumber$ vero -- questa proprietà è la proprietà
    precedente negata;
  \item $Livelock$ falso -- non sono presenti Livelock nel presente, dal
    momento che non sono previsti \texttt{NACK} in questa implementazione.
\end{itemize}

\paragraph{UnreliableABP} \mbox{}

Per il sistema lossy vengono provate le seguenti proprietà:

\begin{itemize}
  \item $Even(<ok_0>T)$ falso -- non è vero che in tutte le computazioni
    complete venga ricevuto il messaggio con numero di sequenza 0, siccome il
    sistema potrebbe rimanere in ciclo infinito. \\
    In pratica, non vale la \emph{fair abstraction from divergence};
  \item $Even(<ok_1>T)$ falso -- non è vero che in tutte le computazioni
    complete venga ricevuto il messaggio con numero di sequenza 1, siccome il
    sistema potrebbe rimanere in ciclo infinito. \\
    In pratica, non vale la \emph{fair abstraction from divergence};
  \item $Fair$: vero -- è invariante che in qualsiasi modo venga eseguito $ok$
    per un certo messaggio di sequenza, poi verrà eseguito quello per il
    messaggio con il numero di sequenza successivo;
  \item $NoConsecutiveSameNumber$ vero -- il \textbf{BadReceiver} ignora i
    messaggi aventi numeri di sequenza uguali a quello ricevuto correttamente;
  \item $Livelock$ vero -- il \textbf{BadReceiver} potrebbe continuare aventi
    rifiutare un certo messaggio scegliendo di inviare sempre dei \texttt{NACK}
    all'\textbf{AckReceiver}.
\end{itemize}

\paragraph{LoudReliableABP} \mbox{}

Per l'implementazione estesa del protocollo con canale affidabile vengono
provate:

\begin{itemize}
  \item $LoudOk_0$ vero -- Ogniqualvolta un sistema manda un messaggio con
    numero di sequenza 0 (e dunque verrà effettuata una comunicazione sul
    canale $send_0$), questo verrà ricevuto (verrà effettuata una
    comunicazione sul canale $receive_0$);
  \item $LoudOk_1$ vero -- Ogniqualvolta un sistema manda un messaggio con
    numero di sequenza 1 (e dunque verrà effettuata una comunicazione sul
    canale $send_1$), questo verrà ricevuto (verrà effettuata una
    comunicazione sul canale $receive_1$);
  \item $LoudOk$ vero -- vero poichè entrambe le proprietà sopra descritte
    sono vere.
\end{itemize}

\paragraph{LoudUnreliableABP} \mbox{}

Per l'implementazione estesa del protocollo con canale affidabile vengono
provate:

\begin{itemize}
  \item $LoudOk_0$ falso -- Ogniqualvolta un sistema manda un messaggio con
    numero di sequenza 0 (e dunque verrà effettuata una comunicazione sul
    canale $send_0$), il canale ricevente potrebbe non accettarlo mai;
  \item $LoudOk_1$ falso -- Ogniqualvolta un sistema manda un messaggio con
    numero di sequenza 1 (e dunque verrà effettuata una comunicazione sul
    canale $send_1$), il canale ricevente potrebbe non accettarlo mai;
  \item $LoudOk$ falso -- falso poichè entrambe le proprietà sopra descritte
    sono false.
\end{itemize}
