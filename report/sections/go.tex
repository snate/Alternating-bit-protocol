\newpage{}\section{Go}

\subsection{Perchè Go}\label{sec:why-do-we-go}

Google Go è un linguaggio di programmazione che basa i suoi principi sulla
gestione della concorrenza tramite comunicazione, mentre gran parte degli
altri linguaggi di programmazione utilizzano un approccio più vulnerabile,
condividendo memoria.

Condividendo canali di comunicazione, invece, è possibile realizzare programmi
robusti che riescano a soddisfare le stesse proprietà provate teoricamente con
strumenti come il CWB.

\subsection{Implementazione}

Il programma è stato implementato con le due versioni viste in precedenza,
ovvero quella che simula un canale affidabile (sez. \ref{sec:abp-prog}) e
quella che simula un canale con perdite (sez. \ref{sec:bad-abp-prog}).

I due programmi sono stati scritti tenendo conto che, come detto nella sezione
\ref{sec:why-do-we-go}, il paradigma pensato dai creatori di questo linguaggio
consisteva nel gestire la concorrenza tramite comunicazione e non condivisione
di memoria.

Questo ha permesso di riutilizzare gli script usati per definire gli agenti
usati nel CWB e trasformarli in pochissimo tempo in programmi equivalenti in
Go.

Ciascun agente è stato trasformato in una \emph{goroutine} e lanciate nel
metodo main, così come i canali condivisi vengono istanziati nel main e
passati esclusivamente a chi comunica su quei canali.
